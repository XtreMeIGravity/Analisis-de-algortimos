\documentclass[12pt,twoside]{article}
\usepackage{amsmath, amssymb}
\usepackage{amsmath}
\usepackage[active]{srcltx}
\usepackage{amssymb}
\usepackage{amscd}
\usepackage{makeidx}
\usepackage{amsthm}
\usepackage{algpseudocode}
\usepackage{algorithm}
\renewcommand{\baselinestretch}{1}
\setcounter{page}{1}
\setlength{\textheight}{21.6cm}
\setlength{\textwidth}{14cm}
\setlength{\oddsidemargin}{1cm}
\setlength{\evensidemargin}{1cm}
\pagestyle{myheadings}
\thispagestyle{empty}
\markboth{\small{Pr\'actica 1. \'Unicamente Nombre Alumno 1, \'Unicamente Nombre Alumno
2.}}{\small{.}}
\date{}
\begin{document}
\centerline{\bf An\'alisis de Algoritmos, Sem: 2020-1, 3CVX, Pr\'actica 1, Fecha}
\centerline{}
\centerline{}
\begin{center}
\Large{\textsc{Pr\'actica 1: Determinaci\'on experimental de la complejidad temporal de un
algoritmo}}
\end{center}
\centerline{}
\centerline{\bf {Nombre Completo Alumno 1, Nombre Completo Alumno 2.}}
\centerline{}
\centerline{Escuela Superior de C\'omputo}
\centerline{Instituto Polit\'ecnico Nacional, M\'exico}
\centerline{$correo@alumno_1, correo@alumno_2$}
\newtheorem{Theorem}{\quad Theorem}[section]
\newtheorem{Definition}[Theorem]{\quad Definition}
\newtheorem{Corollary}[Theorem]{\quad Corollary}
\newtheorem{Lemma}[Theorem]{\quad Lemma}
\newtheorem{Example}[Theorem]{\quad Example}
\bigskip
\textbf{Resumen:} Redactar de manera breve y concisa de que trata el trabajo presentado. Un
s\'olo p\'arrafo.
{\bf Palabras Clave:} Colocar de 3 a 5 palabras clave.
\section{Introducci\'on}
En est\'a secci\'on, como su nombre lo indica, introducir al lector al trabajo presentado. En el
caso de esta primera pr\'actica podr\'ian comenzar explicando la importancia de los algoritmos y
la importancia de analizar un algoritmo, y finalizar con el objetivo de la pr\'actica.
Indicaciones para el env\'io de la pr\'actica: 1) Tienen a los m\'as 8 d\'ias para enviar sus
pr\'acticas a la cuenta de correo que se les proporcionara. 2) En "Asunto" del correo escribir 
el n\'umero de pr\'actica (Pr\'actica No. 1), y en el contenido del correo escribir los nombres
completos de los integrantes del equipo (2 personas a lo m\'as) y si tienen alg\'un comentario
respecto a la pr\'actica. 3) En cada Pr\'actica se tiene que enviar todo el c\'odigo fuente de
sus programas y en el contenido del correo especificar en que lenguajes fueron realizados.
Adem\'as enviar el reporte en PDF y su respectivo c\'odigo en LaTeX. 4) Recuerden que los
c\'odigos de sus algoritmos implementados deben de tener plantilla de datos y comentarios.
\section{Conceptos B\'asicos}
Aqu\'i va todo lo necesario para comprender el trabajo. En el caso de esta pr\'actica podr\'ian
colocar los conceptos de $\Theta$, $O$ y $\Omega$. Adem\'as comentar que algoritmos se
desarrollar\'an y mostrar los algoritmos, pueden mostrar ejemplos del funcionamiento del
algoritmo implementado.
\section{Experimentaci\'on y Resultados}
En est\'a secci\'on tiene que ir toda la experimentaci\'on que hayan hecho referente a la
pr\'actica. Aqu\'i van impresiones de pantalla del programa en ejecuci\'on y su explicaci\'on,
gr\'aficas y su explicaci\'on, adem\'as va toda la explicaci\'on de lo que se pide en la
pr\'actica.
\section{Conclusiones}
Las conclusiones de manera general y de manera individual. En lo general, podr\'ian escribir,
errores que se presentaron y como se resolvieron, podr\'ian escribir observaciones de como
mejorar el algoritmo, si quedaron los resultados esperados o no, y por que, etc.
Las conclusiones individuales ya cada quie  que redactar. A las conclusiones individuales
se les anexar\'a una fotografía del cada alumno.
Conclusiones Alumno 1 (FOTO)
Conclusiones ALumno 2 (FOTO)
\section{Anexo}
Si se dejan problemas para resolver, en esta secci\'on se mostrar\'an.
\section{Bibliograf\'ia}
Mostrar referencias en formato IEEE.
\end{document}