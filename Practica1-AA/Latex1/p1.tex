\documentclass[12pt,twoside]{article}
\usepackage{amsmath, amssymb}
\usepackage{amsmath}
\usepackage[active]{srcltx}
\usepackage{amssymb}
\usepackage{amscd}
\usepackage{makeidx}
\usepackage{amsthm}
\usepackage{algpseudocode}
\usepackage{verbatim}
\usepackage{algorithm}
\usepackage{graphicx}
\usepackage[utf8]{inputenc}
\renewcommand{\baselinestretch}{1}
\setcounter{page}{1}
\setlength{\textheight}{21.6cm}
\setlength{\parskip}{0cm plus 5mm minus 3mm}
\setlength{\textwidth}{14cm}
\setlength{\oddsidemargin}{1cm}
\setlength{\evensidemargin}{1cm}
\pagestyle{myheadings}
\thispagestyle{empty}
\markboth{\small{Pr\'actica 1. David, Jazmin }}{\small{.}}
\date{}
\begin{document}
	\centerline{\bf An\'alisis de Algoritmos, Sem: 2020-2, 3CV2, Pr\'actica 1, 29/01/2020}
	\begin{center}
		\Large{\textsc{\\ Pr\'actica 1: Determinaci\'on experimental de la complejidad temporal de un algoritmo.}}
	\end{center}
	\centerline{}
	\centerline{\bf {Nombres: L\'opez Hern\'andez David, Tolentino P\'erez Jazmin Yaneli }}
    \centerline{}
	\centerline{Escuela Superior de C\'omputo}
	\centerline{Instituto Polit\'ecnico Nacional, M\'exico}
	\centerline{$xtremeigraviti@hotmail.com ,  jaz\_01\_99@hotmail.com$}
    \centerline{}
	\textbf{Resumen:}\ Durante la práctica se desarrollarón dos algoritmos, el primero que realiza la suma binaria de dos números de diferentes dimensiones y uno más que busca el mcd de dos números consecutivos en la serie de Fibonacci. \\ 
	\centerline{}
	\textbf{Palabras clave:}\ Iterativo, Complejidad
	\section{Introducci\'on}
	
	\section{Conceptos B\'asicos}
	
	\section{Experimentaci\'on y Resultados}
	
	\section{Conclusiones}
	
	\section{Anexo}
	
	\section{Bibliograf\'ia}
	
\end{document}

